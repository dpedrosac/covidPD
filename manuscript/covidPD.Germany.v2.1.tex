\documentclass[a4paper,oneside,11pt,english]{scrartcl} 

%% Preamble
\usepackage[T1]{fontenc}
\usepackage[utf8]{inputenc}
\usepackage[english]{babel}
\selectlanguage{english}

\usepackage[onehalfspacing]{setspace}
\usepackage[nohyperlinks, printonlyused, withpage, smaller]{acronym}
\usepackage{paralist}
\usepackage{graphicx}
\graphicspath{ {media/} }

\usepackage[hyphens]{url}
\urlstyle{same}

\usepackage{subcaption}
\captionsetup[subfigure]{width=0.9\textwidth}

\usepackage{booktabs}
\usepackage{multirow}
\usepackage{makecell}
\usepackage{rotating}
\usepackage{longtable}

% References and bibliography
\usepackage[backend=biber, style=numeric-comp, giveninits=true, maxnames=5, minnames=3, bibencoding=utf8, sorting=none]{biblatex}
\addbibresource{bmc_article_covidPD.bib}
\AtEveryBibitem{%
  \clearfield{number}}
\renewcommand*{\newunitpunct}{\addcomma\space}
\renewbibmacro{in:}{}

% Change bibliography according to J Parkinson Disease template
\newbibmacro*{journal+issuetitle}{%
  \setunit{\addcolon\space}%
  \usebibmacro{journal}%
  \setunit*{\addspace}%
  \usebibmacro{volume+number+eid}%
  \setunit{\addspace}%
  \newunit}

\DeclareBibliographyDriver{article}{%
  \usebibmacro{bibindex}%
  \usebibmacro{begentry}%
  \usebibmacro{author/translator+others}%
  \setunit{\labelnamepunct}\newblock
    \setunit{\addspace}% NEW
  \usebibmacro{issue+date}% NEW
  \setunit{\addcolon\space}% NEW
  \usebibmacro{title}%
  \newunit
  \printlist{language}%
  \newunit\newblock
  \usebibmacro{byauthor}%
  \newunit\newblock
  \usebibmacro{bytranslator+others}%
  \newunit\newblock
  \printfield{version}%
  \newunit\newblock
  \usebibmacro{journal+issuetitle}%
  \newunit
  \usebibmacro{byeditor+others}%
  \setunit{\addspace}% NEW
  \newunit\newblock
  \usebibmacro{finentry}}

\DeclareNameAlias{sortname}{last-first}
\DeclareFieldFormat[article,inbook,incollection,inproceedings,patent,thesis,unpublished]{title}{\mkbibquote{#1\isdot}}
\DeclareFieldFormat[article,inbook,incollection,inproceedings,patent,thesis,unpublished]{volume}{\addcomma\space\textbf{#1}}
\DeclareFieldFormat[article]{pages}{#1}

\begin{document}
\begin{titlepage}
\noindent\LARGE{\textbf{Impact of the \textsc{COVID}-19 Pandemic on Perceived Access and Quality of Care in German People with Parkinson's Disease}}
	
\section*{Authors:}
\large{Marlena van Munster$^{1}$, Marcel R. Printz$^{1}$, Eric Crighton$^{2}$, Tiago A. Mestre$^{3}$, David J. Pedrosa$^{1,4,*}$ and iCARE-PD consortium}\\
	
\begin{compactenum}[$^\bgroup1\egroup$] 
\item \small{Department of Neurology, Philipps University Marburg, Baldingerstraße, 35043 Marburg, Germany}
\item \small{Department of Geography, Environment and Geomatics, University of Ottawa. Simard Hall, 60 University Private,  Ottawa, Ontario, Canada}
\item \small{Parkinson’s Disease and Movement Disorders Clinic, Division of Neurology. Department of Medicine. The Ottawa Hospital Research Institute, The University of Ottawa Brain and Mind Research Institute, Ottawa, ON, Canada}
\item \small{Centre of Mind, Brain and Behaviour, Philipps University Marburg, Hans Meerwein Straße 6, 35032 Marburg, Germany}\\
\end{compactenum}

\noindent\small{$^{*}$ correspondence: Department of Neurology, Philipps University Marburg, Baldingerstraße, 35043 Marburg, Germany - david.pedrosa@staff.uni-marburg.de}
\end{titlepage}


\section*{Abstract}
\subsection*{Background}
Due to the heterogeneous clinical presentation, people with Parkinson's disease (PwPD) develop individual health care needs as their disease progresses. However, as a result of limited health resources during the \textsc{COVID}-19 pandemic, many patients were put at risk of inadequate care. All this occured in the context of inequitable health care provision within societies, especially for such vulnerable populations.
\subsection*{Objective}
This study aimed to investigate factors influencing satisfaction and unmet need for health care among Pw\textsc{PD} during the \textsc{COVID}-19 pandemic in Germany.
\subsection*{Methods} 
Analyses relied on an anonymous online survey with a 49-item questionnaire. We aimed at describing access to health services before and during early stages of the pandemic. To this end, a Generalized Linear Model (GLM) was used to derive significant predictors and a stepwise regression to subsummarise the main factors of perceived inadequate care.  
\subsection*{Results}
In total, 551 questionnaires showed that satisfaction for \textsc{PD}-related care decreased significantly during the pandemic (\textit{p} $<$ .001). Especially, factors such as lower educational level, lower perceived expertise of healthcare providers, less confidence in remote care, difficulties in obtaining healthcare and restricted access to care prior to the pandemic but also lower densites of neurologists at residence and less ability to overcome barriers were indicative of higher odds to perceive unmet needs (\textit{p} $<$ .05).
\subsection*{Conclusion}
The results unveil obstacles contributing to reduced access to healthcare during the \textsc{COVID}-19 pandemic for PwPD. These findings enable considerations for improved provision of healthcare services to PwPD.
\subsubsection*{Keywords}
\small{Parkinson's disease; \textsc{COVID}-19 pandemic; health care; impact; Germany; access}

\newpage

\section*{Introduction}

The \textsc{COVID}-19 pandemic presented unprecedented challenges worldwide afflicting people adversely economically and culturally. In response to rising caseloads, public life was shut down and access to health services, among others, was disrupted. \cite{nunez2021access, moynihan2021impact, world2020impact}. Scientific evidence suggests that health care utilization declined by about one-third during the pandemic \cite{moynihan2021impact}. In Germany, a decrease in the use of outpatient and inpatient services was reported during the first wave, with dental and specialist examinations being cancelled most frequently, followed by physiotherapy, occupational therapy or speech therapy. \cite{Heidemann2022Non-utilisation}. Yet, this disruption affected individuals in Germany to varying degrees and especially those with chronic diseases, such as persons with Parkinson's disease (Pw\textsc{PD}) \cite{kasar2021life,yogev2021covid,scheidt2021care, sepulveda2020impact}. This is not too surprising in that Pw\textsc{PD} belong to the high-risk group for severe disease or for secondary complications of \textsc{COVID}-19, which made them reluctant to visit medical facilities \cite{feral2020collateral}. 

Pw\textsc{PD} show a progressive condition characterized by motor but also non-motor symptoms. A plethora of different clinical signs may emerge during the disease's course, requiring continuous therapy adjustments and need assessments by healthcare professionals. \textsc{PD} negatively affects individual psychosocial functioning \cite{demirtepe2022psychosocial}, often leaving those affected in need of social, financial or physical support. People suffering from chronic diseases, including Pw\textsc{PD}, often necessitate continuous medical services outside of emergency departments, such as frequent physiotherapy, and therefore appeared at high risk of undersupply during the pandemic \cite{scheidt2021care, nunez2021access, world2020impact}. Recent studies have unveiled the impact of the \textsc{COVID}-19 pandemic on people suffering from \textsc{PD} \cite{yogev2021covid, zipprich2020knowledge, frundt2022impact, richter2021analysis, brooks2021social}. For the German population, Zipprich et al. interviewed Pw\textsc{PD} about their experience of healthcare during the pandemic. About one-third indicated that they experienced a decrease in their mobility because regular therapies (e.g., physiotherapy) were cancelled \cite{zipprich2020knowledge}. Fründt et al. also showed that Pw\textsc{PD} who received long-term care were more socially isolated during the pandemic than those who did not receive long-term care. Thus, it seems likely that Pw\textsc{PD} were affected to varying degrees by the constraints during the pandemic, not least because other areas of public health research also suggest that health crises have a highly individualized impact on access to care for vulnerable groups \cite{huijts2017prevalence, lowcock2012social, whocovidbrief}.

Beyond the variable degree of disability due to \textsc{PD} and the resulting and highly individual needs other determinants influencing how severely care is restricted may be inferred. Thus, determinants of access to healthcare may pose an interesting concept to answer the question of what is relevant to maintain a high level of support and well-being individually but also on a societal level. What can be considered a relevant determinant, however, is by no means universal and rather context-specific considerations are required \cite{world2010conceptual}. For \textsc{PD}, Zaman et al. proposed a model summarising structural and individual factors potentially influencing patients' access to healthcare \cite{zaman2021barriers}. Structural determinants may on the one hand encompass barriers, that \textsc{PD}-patients meet on a system-level when accessing healthcare, such as a lack of care coordination, limited communication between healthcare providers, disparities in health services or the unavailability of specialised services \cite{zaman2021barriers}, etc. Otherwise, individual barriers influencing \textsc{PD}-patients' abilities to seek help or to engage with care providers, to reach important care services or to pay for them \cite{zaman2021barriers} may likewise be of great importance. Particularly people with \textsc{PD} often hinge on a good support network. 

To our knowledge, it has not yet been investigated how determinants of access to healthcare may relate to the perceived healthcare situation during the \textsc{COVID}-19 pandemic of Pw\textsc{PD} in Germany. Therefore, we examined the impact of a multitude of factors on this population with special emphasis on their access to healthcare.

\newpage
%%%%%%%%%%%%%%%%
%% Methods    %%
%%%%%%%%%%%%%%%%

\section*{Methods}
We conducted a cross-sectional survey of Pw\textsc{PD} in Germany (or their caregivers). Participants at all stages of the disease were eligible to participate in the survey which consisted of a anonymous questionnaire. This questionnaire was distributed nationwide using the members' e-mail newsletter of the German Parkinson Association (Deutsche Parkinson Vereinigung e.V., dPV) between November 2020 to January 2021. The e-mail included a short invitation with a link to an online survey, which patients could access using a personal computer, a tablet or a smartphone. In Germany, SoSci Survey \cite{leiner2019sosci} served as a database for hosting the survey. Throughout the data input, the database was supervised and manually checked for plausibility. The study was approved by the local Ethics committee (reference number: AZ 164/19) and carried out in accordance with the Declaration of Helsinki. All patients gave informed written consent prior to participating.

\subsection*{Questionnaire}
This work was carried out as part of the multinational iCARE-\textsc{PD}-project (\url{https://icare-pd.ca/}). Within the scope of this project, a 49-item questionnaire was developed which aimed at characterising the access of Pw\textsc{PD} to healthcare services before and during the pandemic. In addition to Germany, the iCARE-\textsc{PD} questionnaire was also shared with patient associations in Canada, Spain, Portugal and the Czech Republic with the respective translations. In this study, we limited ourselves to data collected from German patients. Fur that purpose, the initial questions in English were translated to German and were structured in four sections: A) patients' health status in terms of \textsc{PD}, operationalized by \cite{hoenn1967parkinsonism} and \cite{jenkinson1997pdq}, but also  concomitant diseases, B) experiences with healthcare services within twelve months before the pandemic, C) experiences with healthcare services during the \textsc{COVID}-19 pandemic with special emphasis on telemedicine services, and, D) demographic and socioeconomic characteristics of participants. There were single and multiple-choice questions along with open-ended questions, some of which depended upon the specific answers to previous ones. A full version of the questionnaire is included in the supplementary data. 

\subsection*{Statistical analyses}
All analyses were conducted in R \cite{rcore}. Publicly available data on population densities \footnote{\url{https://www.bbsr.bund.de/BBSR/DE/forschung/raumbeobachtung/Raumabgrenzungen/deutschland/regionen/Raumordnungsregionen/raumordnungsregionen-2017.xlsx?\_\_blob=publicationFile\&v=3}} and those for neurologists\footnote{\url{https://gesundheitsdaten.kbv.de/cms/html/16402.php}} could be added to the analyses for regional data containment. For that purpose, we used the first three numbers of their German postal code, which were disclosed in the last section of the survey. Merging the available data with the maps for postal codes\footnote{\url{https://www.suche-postleitzahl.org/downloads}} resulted in regional distributions (cf. Figure \ref{fig1:total}). Population and neurologist densities were stratified into five equal quantiles for further analyses. Moreover, the provided information of concomitant diseases (besides \textsc{PD}) was collated to a score -- the Elixhäuser Comorbitiy Score with its modification introduced by van Walraven et al. \cite{van2009modification} with higher values indicating more severe disease burden. Finally, all questions were assigned to barriers to accessing health services regarding \textsc{PD} as described by \cite{zaman2021barriers} (cf. Table \ref{tab3:matchingzaman} in the supplementary data).

\begin{figure}[h!]
	\centering
	\begin{subfigure}[b]{0.35\linewidth}
		\includegraphics[width=.90\textwidth]{fig1a.questionnaires.v1.0.png}
		\label{fig1:questionnaires}
	\end{subfigure}%
	\begin{subfigure}[b]{0.35\linewidth}
		\includegraphics[width=.90\textwidth]{fig1b.population_density.1.0.png}
		\label{fig1:population}
	\end{subfigure}%
	\begin{subfigure}[b]{0.35\linewidth}
		\includegraphics[width=.90\textwidth]{fig1c.neurologist_density.v1.0.png}
		\label{fig1:neurologists}
	\end{subfigure}%
	\caption{Demographic data for Germany and additional regional data for the obtained questionnaires. A) Number of received questionnaires within our survey for the distinct three digit postal codes. B) Illustration of inhabitants per square kilometer for Germany. C) Density of neurologists in all parts of Germany according to the German Statutory Health Insurance Association (\textit{Kassenärztliche Bundesvereinigung})}
	\label{fig1:total}
\end{figure}

After estimation of descriptive statistics, satisfaction with overall \textsc{PD}-related care was compared before and during the pandemic using a non-parametric \textit{sign-test} (rstatix package, \url{https://github.com/kassambara/rstatix/}). The two questions that were used were: 
\begin{itemize}
	\item ``In the 12 months prior to the  \textsc{COVID}-19 pandemic, overall, how satisfied were you with the way healthcare services related to \textsc{PD} were provided?'' (B17) vs.
	\item ``Since the beginning of the \textsc{COVID}-19 pandemic, overall, how satisfied are you with the way healthcare services related to \textsc{PD} are provided?'' (C6).
\end{itemize}

Furthermore, using a Generalized Linear Model (\textsc{GLM}) with a binomial link function, we estimated Odds ratios for worse satisfaction with \textsc{PD}-related care. After establishing the full model with a total of 32 predictors, we conducted a stepwise logistic regression in order to reduce the complexity, leaving the most meaningful predictors for the question: ``Since the beginning of the \textsc{COVID}-19 pandemic, how often did you feel you needed healthcare for \textsc{PD} but did not receive it?'' (C4). For that, first missing data were imputed by taking advantage of a multivariate imputation scheme using the \textsc{Mice}-package \cite{vanbuuren2011}. We thereby assumed data missing at random and used the Predictive Mean Matching Method. Consecutively, stepwise reduction using a \textsc{GLM} with Stepwise Feature Selection (\textit{glmStepAIC}) in both directions from the \textit{caret}-package \cite{kuhn2008} aimed at minimising the Akaike Information Criterion (\textit{AIC}). We first split all available data into 80\% of training and 20\% of test data and performed the stepwise regression after centering and rescaling values and by applying 10-fold cross-validation. The predictions of the two models were compared with the test data using accuracy, Area Under the Curve (\textsc{AUC}) and LogLoss as metrics. All data and analyses are available at \url{https://github.com/dpedrosac/covidPD/}

\newpage

%%%%%%%%%%%%%%%%
%% Results %%
%%%%%%%%%%%%%%%%

\section*{Results}
In total, 551 questionnaires (response rate about 3\%) were filled out with 252 different postal codes from all 16 German regions (Bundesländer, cf. Figure~\ref{fig1:total}). Of all participants, 388 (70.4$\%$) returned a complete questionnaire (for demographics from parts A and D, cf. Table~\ref{table1}).

\begin{table}[h!]
	\caption{Demographics and clinical characteristics of survey respondents}
	\label{table1}
	\begin{tabular}{p{10cm} c}
		\toprule
		&\textbf{Overall}\\ %\hline
		& \textbf{(n = 551)}\\ \hline
		
		Age (mean (SD)) & 66.76 (9.25) \\ \hline
		Gender = Female (\%) &  148 (41.6)  \\ \hline
		Time since PD diagnosis (\%) & \\ \hline
		\hspace{3mm} $<$2 years & 62 (13.1) \\ \hline
		\hspace{3mm} 2--5 years & 154 (32.6) \\ \hline
		\hspace{3mm} 5--10 years & 157 (33.2) \\ \hline
		\hspace{3mm} 10--15 years & 69 (14.6) \\ \hline
		\hspace{3mm} $>$15 years& 31 ( 6.6) \\ \hline
		Disease stage (\%)& \\ \hline
		\hspace{3mm} Hoehn \& Yahr I &  189 (40.3) \\ \hline
		\hspace{3mm} Hoehn \& Yahr II & 156 (33.3)  \\ \hline
		\hspace{3mm} Hoehn \& Yahr III  &   77 (16.4) \\ \hline
		\hspace{3mm} Hoehn \& Yahr IV  & 41 ( 8.7) \\ \hline
		\hspace{3mm} Hoehn \& Yahr V  &     6 ( 1.3) \\ \hline
		Education level according to ISCED (\%) & \\ \hline
		\hspace{3mm} primary education  & 20 ( 5.0) \\ \hline
		\hspace{3mm} secondary education  & 234 (58.4)\\ \hline
		\hspace{3mm} post secondary education  &   69 (17.2) \\ \hline
		\hspace{3mm} highest education level possible & 78 (19.5)  \\ \hline
		\hspace{3mm} \textsc{PD}Q-8 scores (mean (SD)) & 41.30 (14.23) \\ \hline
		Van-Walraven-Elixhauser Comorbidity Index (mean (SD)) & 6.55 (1.95) \\ \hline
		\bottomrule
	\end{tabular}
\end{table}

Satisfaction for \textsc{PD}-related care significantly decreased during the pandemic (pre-pandemic, Mdn = 3 vs. post-pandemic, Mdn = 1; \textit{p} = 10\textsuperscript{-73}). More than 90\% of all participants stated to be somewhat unsatisfied or very unsatisfied with their \textsc{PD}-related care during the pandemic (cf. Figure \ref{fig2:satisfaction}). 

\begin{figure}[h!]
	\centering
	\includegraphics[width=.90\textwidth]{fig2.satisfaction.care.v1.0.jpeg}
	\caption{B17 vs. C6 - Distribution of responses on the Satisfaction with PD-related care before and during the \textsc{COVID}-19 pandemic.}
	\label{fig2:satisfaction}
\end{figure}

To ascertain factors associated with declines in satisfaction, logistic regressions on question C4 (``Since the beginning of the \textsc{COVID}-19 pandemic, how often did you feel you needed healthcare for \textsc{PD} but did not receive it?'') was performed, unveiling factors which contributing to this perception of unmet needs during the pandemic (see \mbox{Figure \ref{fig3:resultsOR1}}). 

\begin{figure}[!h]
	%Kann man die Grafik größer anzeigen und vielleicht sortieren?
	%% DP: Wonach sortieren? Die Grafiken werden am Ende nicht eingefügt sondern als svg mitgeschickt, das Layout dürfen die übernehmen ;) 
	\centering
	\includegraphics[width=\textwidth]{fig3.oddsratios.v1.0.jpeg}
	\caption{Odds ratios for all items in terms of perceived inadequate healthcare during pandemic. Odds were determined via \textsc{GLM} and coded so that higher values indicate affirmation to the question that healthcare was needed but this need remianed unmet during the \textsc{COVID}-19 pandemic. The dashed lines indicate the distinct domains according to Zaman et al. \cite{zaman2021barriers}, whereas significance is illustrated as color of the dot, with two distinct levels of significance. }
	\label{fig3:resultsOR1}
\end{figure}

Thus, odds to affirm this question were highly significant (\textit{p} $<$ .001) for those patients inferring lower levels of competence for their neurologist, with a lower ability to access \textsc{PD}-care before the pandemic, for patients with higher degrees of stigmatisation in healthcare but also for those who did not receive healthcare services before the pandemic. A significant contribution -- albeit lower with significance values \textit{p} $<$ .05 --  was encountered for Pw\textsc{PD} with increasing levels of comorbidity, with perceived lower expertise of the general practitioner, with higher quality of life scores retrospectively, for people with higher financial burden due to \textsc{PD} or who needed to reschedule healthcare appointments due to financial problems before the pandemic. Finally, the lack of availability of remote healthcare during the pandemic and geographical or in general more numerous barriers in access to healthcare before the start of the pandemic were also indicative of higher odds to perceive unmet needs. For an illustration of significant predictors see Figure \ref{fig3:resultsOR1} and for the entire list of results cf. Table \ref{tab4:resultsall1} in the supplementary material. In assumption of an overfitted model, we performed a two-way stepwise regression for question ``C4'' (see above) so that the intitial 32 items could be reduced to seven significant predictors of unmet needs for healthcare services (cf. Table \ref{tab2:reduced_model}), namely: 
\begin{itemize}
	\item[--] educational level
	\item[--] perceived expertise of the general practitioner
	\item[--] confidence in the ability to access required health care services remotely
	\item[--] perceived ease of obtaining healthcare before the pandemic
	\item[--] perceived availability of specialist care before the pandemic
	\item[--] density of neurologists within the area of living
	\item[--] availability of structural support to overcome geographical barriers
\end{itemize}

\begin{table}[ht]
	\caption{Significant factors contibuting to unmet care needs during \textsc{COVID}-19 pandemic according to the reduced \textsc{GLM}:}
	\label{tab2:reduced_model}
	\centering
	\resizebox{\columnwidth}{!}{\begin{tabular}{l c c c c}
			\toprule   
			\textbf{Predictor}	& \textbf{Estimate}					& \textbf{SE} & \textbf{\textit{z}-value} & \textbf{\textit{p}} \\ \hline
			(Intercept) 											& -2.65  		& 0.29 	& -9.24 	& $<$.0001 \\ \hline
			Educational level (D8) 									& -0.73 		& 0.24 	& -3.01 	& 0.003 \\ \hline
			Perceived GP's expertise (B3) 							& 0.34 		& 0.17 	& 2.07 	& 0.038 \\ \hline
			Confidence in accessing necessary services remotely (C3) 	& 0.64 		& 0.22 	& 2.90 	& 0.004 \\ \hline
			Ease obtaining healthcare prior to the pandemic (B10)		& -0.47 		& 0.22 	& -2.15 	& 0.031 \\ \hline
			Ability to access care prior to the pandemic (B9)				& 0.41 		& 0.20 	& 2.07 	& 0.038 \\ \hline
			Density of Neurologists 									& 0.47 		& 0.21 	& 2.22 	& 0.027 \\ \hline
			Overcoming barriers (B7a)   								& -0.51 		& 0.22 	& -2.38 	& 0.017 \\ \hline
			\bottomrule
	\end{tabular}}
\end{table}	 

Markers for model comparison were indicative of similar performances in the ``full model'' with 32 predictors compared to the reduced one (cf. Figure \ref{fig4:comparison_models})

%Figure 4 hier einbinden

\newpage

%%%%%%%%%%%%%%%%
%% Discussion %%
%%%%%%%%%%%%%%%%

\section*{Discussion}
In this study, we identified factors such as lower educational levels, a lack of perceived expertise in the treating physicians and structural obstacles or lack of support offerings as important factors contributing to insecurity and the feeling of not having received adequate health services during the \textsc{COVID}-19 pandemic among German Pw\textsc{PD}. To the best of our knowledge, this is the first time that determinants of \textsc{PD}-patients' perceived access to healthcare were investigated. With our study, we demonstrate that not all individuals were affected equally but that structural as well as individual determinants infer perceived access to healthcare. Viewing the pandemic through the focal lens of an ongoing demographic change in Western societies, our findings may render a deeper insight into how future care of Pw\textsc{PD} may be improved. 

Our results substantiate that structural challenges for individuals with \textsc{PD} reinforce perceived insecurity and a feeling of not obtaining the needed healthcare. The majority of predictors from the reduced model and eight predictors from the full model may be projected to system-level ``barrier'' put forward by Zaman et al. Interestingly, a good overall performance has been attested to the German healthcare system during the pandemic \cite{10665-341674}, which is transferable to Pw\textsc{PD} \cite{frundt2022impact}. However, a good testimony for a healthcare system should not be equated with an adequate range of services, especially when it comes to very specific needs of \textsc{PD}-patients. In the recent literature, care deficits on a structural level have been reported insinuating a rather partial insufficiency \cite{richter2019dynamics,van2020building, prell2020specialized, deuschl2016s3} for this heterogeneous population. One of the major challenges physicians face when treating \textsc{PD} is its diverse clinical manifestation. Multimodal complex treatments could be a potential remedy \cite{richter2019dynamics}, yet limited availability of such services cause long journeys for people from some regions \cite{richter2019dynamics} as coordinated care approaches for Pw\textsc{PD} remain rare in some parts of Germany \cite{van2020building}. Furthermore, staff providing specialised, structured and cooperative care services are lacking especially in outpatient care and in nursing homes \cite{prell2020specialized} despite being advisable \cite{radder2020recommendations, deuschl2016s3}.

On the level of individual determinants, our data may also have some implications. We identified low educational attainment as a predictor for the perception of inadequate healthcare, which according to Zaman \cite{zaman2021barriers} relates to two dimensions: health literacy and self-efficacy. The former inversely correlates with the ability to express healthcare needs \cite{davis2003variability, hurt2019barriers} and with educational levels of Pw\textsc{PD} \cite{fleisher2016associations}. This is in good accordance with higher rates of hospitalisations and a higher caregiver burden \cite{fleisher2016associations} as well as higher disease severity \cite{fleisher2016associations} in Pw\textsc{PD} with lower health literacy. With regards to higher self-efficacy, this correlates with the level of education\cite{lim2020factors} and, at the same time, with quality of life \cite{rostagni2022gratitude, lim2020factors}. Hence, our model suggests that Pw\textsc{PD} who have received more education and who present with higher quality of life scores show the greatest probability to absorb disruptions in healthcare. Contrarily, our data hence advocates for greater attention tp \textsc{PD}-patients with lower levels of education, but particularly those with quality of life restrictions.

Unsurprisingly, Pw\textsc{PD} deemed the expertise of neurologists important on the perception in adequate care. It is well-known that patients' trust in care professionals may foster healthcare utilisation \cite{bainbridge2009challenges, shin2016initiation}. This warrants special emphasis since training appears at first glance accessible. The extent to which a trained Parkinson nurse could facilitate special services and therefore complement medical expertise remains a question to be answered \cite{vanrole}. Yet, one might infer that they could catalyst tailored offerings such as legal or economic counselling. Hence, economic problems were highlighted in our results and are consistently cited as a reason for not seeking care services \cite{zaman2021barriers}. Barring direct costs, e.g. those services spared from health insurance, many patients also claim indirect expenses like those resulting from the inability to work \cite{spottke2005cost}. This may gain importance with increases in the employment of women nowadays.  In general, however, a somewhat surprising result is that women are at higher risk of perceived undersupply. The reasons are unclear, but literature indicates that women have fewer caregivers compared with their spouses \cite{dahodwala2018sex} especially, as they are less likely to receive care from their male partners \cite{zaman2021barriers}. A higher vulnerability to disruptions of healthcare because of the pandemic is therefore feasible and awaits future confirmation. In general, one might posit that to strengthen the resilience of \textsc{PD} health care, strategies are needed that recognize and address both structural and individual barriers in access to healthcare.

In addition to investments, reorganization and policy reforms on the structural level \cite{taylor2016leveraging, gottlieb2019social}, suitable assessments may also help to make the individual needs of patients tangible \cite{friedman2018toward, gottlieb2019social}. One possible solution for subjects lacking access to healthcare services or who may not be able to ask for assistance due insufficient health literacy could be telehealth services. These services are effective means to facilitate access to care. In this questionnaire, we could corroborate this \cite{achey2014past, van2021moving} as \textsc{PD}-patients familiar with telemedicine services before the pandemic reported a reduced likelyhood of unmet care needs. Nevertheless, some caution is advised when interpreting these findings as this cohort must be deemed rather technology-savy according to the nature of the questionnaire. Therefore, this process may not generalisable for all patients \cite{eggers2020care}. Further investigations are warranted, e.g., on how to increase the confidence in telemedicine or how to overcome technological limitations such as high-speed internet availability. Another possible caveat to consider are putative unintended negative effects on health equity, so that Pw\textsc{PD} with low incomes or with other barriers to accessing technology could be left behind \cite{samuels2021digital}. 

\section*{General Limitations:}
At a relatively early stage and before the availability of vaccination provided some relief, our data reflect people's unbiased and acute concerns regarding their own healthcare. Despite revealing problems patients encountered during the \textsc{COVID}-19 pandemic, the interpretation of our results requires some caution. Hence, it was an anonymous online survey, so that the representativeness for the German \textsc{PD}-population is not warranted. The response rate of 3\% of this study was slightly lower than a comparable questionnaire study of \textsc{PD}-clientele \cite{frundt2022impact}. As abovementioned, not only patients filling out the questionnaire may be highly selected from a major support group in Germany. Finally, a limitation is also the fact that there was no way to ascertain misdiagnosis or the correctness of data, so that these results await confirmation in observational studies with controlled demographics.

\section*{Conclusion}
In order to learn from the pandemic in the long term, difficulties in access to healthcare must be uncovered and addressed. The results of this analysis showed that the \textsc{COVID}-19 pandemic did not affect all \textsc{\textsc{PD}}-patients equally, but that people who experienced individual and structural barriers to accessing healthcare before the pandemic were more affected. Therefore, it is important to examine these determinants more closely and to address them in future-oriented, resilient healthcare models. Further investigations into the effect of individual and structural influences as by as Zaman et al. defined on measures of healthcare experiences should be object of further scrutiny. 
\newpage

% \bibliographystyle{plainnat}
%\bibliography{bmc_article_covid\textsc{PD}}      
\printbibliography %[title={References}]

\newpage
\section*{Supplementary Material}
\begin{table}[!ht]
	\settowidth\rotheadsize{Personal-level}
	\caption{Matching of items in the questionnaire to the categories from the work of Zaman et al. \cite{zaman2021barriers}}
	\label{tab3:matchingzaman}
	\centering
	\begin{tabular}{l l l}
		\toprule 
		\textbf{Question from \textsc{COVID}-Survey}& \textbf{Representative for } \\
		\cmidrule{1-2}
		%& \textbf{Person-level Barriers} &	´\\ \hline
		1. A2, B1a, D7 & Autonomy  & \\
		\cmidrule{1-2}
		2. A1, A4, vWEI, B16a & Health Status &\\
		\cmidrule{1-2}
		3. D8 & Health Literacy & \\
		\cmidrule{1-2}
		4. B3, B5 & Health Belief & \multirow[t]{9}{*}{\rothead{\centering\textbf{Person-level Barriers}}} \\
		\cmidrule{1-2}
		5. B14 & Communication (personal) & \\
		\cmidrule{1-2}
		6. \textsc{PD}Q-8 & Self-efficacy &\\
		\cmidrule{1-2}
		7. B7a, B9a/b & Transportation &\\ 
		\cmidrule{1-2}
		8. B11, B12, B13, D9, D10 & Cost of care &\\
		\cmidrule{1-2}
		9. D2 & Other &\\ \hline
		%& \textbf{System-level Barriers} &\\ \hline
		10. NA & Difficulties of Diagnosis & \\
		\cmidrule{1-2}
		11. B6a & Coordination in care &\\
		\cmidrule{1-2}
		12. B15, C2c2 & Communication (system) & \multirow[t]{5}{*}{\rothead{\centering\textbf{System-level barriers}}}\\
		\cmidrule{1-2}
		13. B7, B9b, B10, C3\_3, nPop, nGer & Disparty in Health Services &\\
		\cmidrule{1-2}
		14. B6, B7a, B9, B9a, C2, C2c2 & Unavailability of Specalist Services & \\
		\bottomrule
	\end{tabular}
\end{table}

\newpage
\begin{longtable}[ht!]{|p{5.5cm} | p{3.5cm} | p{1cm} | l | l | p{1.5cm} |}
	\caption{Odds ratios for the distinct items of the questionnaire} 
	\label{tab4:resultsall1} \\ \hline
	\textbf{Factors} & \textbf{Domain} & \textbf{OR} & \textbf{CIlow} & \textbf{CIup} & \textbf{\textit{p}} \\ \hline
	\endhead
	A1 - Disease duration [Years] & Health Status & 0.9 & 0.69 & 1.16 & 0.419 \\ \hline
	A2 - Disease Stage [H\&Y] & Autonomy & 1.13 & 0.87 & 1.47 & 0.367 \\ \hline
	A4 - Presence of comorbidities & Health Status & 1.26 & 1.06 & 1.49 & 0.007 \\ \hline
	B1a - Regular caregiver present & Autonomy & 0.79 & 0.43 & 1.44 & 0.438 \\ \hline
	B3 - Perceived GP expertise & Health Belief & 0.71 & 0.51 & 0.98 & 0.038 \\ \hline
	B5 - Perceived Neurologist expertise & Health Belief & 0.52 & 0.39 & 0.7 & p $<$ .001 \\ \hline
	B6 - No. of healthcare providers consulted pre \textsc{COVID} & Unavailability of Specialitsts Services & 1.24 & 0.77 & 1.99 & 0.374 \\ \hline
	B6a - Perceived cooperation between healthcare providers & Coordination in Care & 0.99 & 0.66 & 1.49 & 0.975 \\ \hline
	B7 - Presence of geographical barriers in access to healthcare pre \textsc{COVID} & Disparty in Health Services & 1.9 & 1.08 & 3.33 & 0.026 \\ \hline
	B7a - No. of structural and transportation ressources against geographical barriers pre \textsc{COVID} & Unavailability of Specialists Services/ Transportation & 2.27 & 0.79 & 6.51 & 0.129 \\ \hline
	B9 - Not received needed healthcare pre \textsc{COVID} & Unavailability of Specialits Services & 2.5 & 1.88 & 3.32 & p $<$ .001 \\ \hline
	B9a - Availability of \textsc{PD}-specific community ressources & Unavailability of Specialists Services/ Transportation & 2.33 & 1.65 & 3.31 & p $<$ .001 \\ \hline
	B9b - No. of structural and transportation barriers in acces to healthcare pre \textsc{COVID} & Disparatiey in Healthcare Services/ Transportation & 1.98 & 1.01 & 3.89 & 0.048 \\ \hline
	B10 - Perceived difficulty of accessing healthcare pre \textsc{COVID} & Disparatiey in Healthcare Services & 5.37 & 2.84 & 10.17 & p $<$ .001 \\ \hline
	B11 - Rescheduled healthcare due to financial burden pre \textsc{COVID} & Cost of care & 2.11 & 1.13 & 3.93 & 0.019 \\ \hline
	B12 - Extended healthcare insurance & Cost of care & 0.83 & 0.46 & 1.48 & 0.521 \\ \hline
	B13 - Financial burden related to \textsc{PD} pre \textsc{COVID} & Cost of care & 2.81 & 1.42 & 5.53 & 0.003 \\ \hline
	B14 - Communication challenges pre \textsc{COVID} & Communication (personal) & 2.63 & 1.18 & 5.82 & 0.017 \\ \hline
	B15 - Experienced stigmatization in healthcare & Communication (system) & 2.84 & 1.6 & 5.03 & p $<$ .001 \\ \hline
	B16a - No. of negative health consequences from barriers to healthcare & Health Status & 1.18 & 0.93 & 1.48 & 0.166 \\ \hline
	C2 - Availability of remote healthcare during \textsc{COVID} & Unavailability of Specialitsts Services & 1.91 & 1.09 & 3.34 & 0.023 \\ \hline
	C2c2 - Access to telehealth technologies during \textsc{COVID} & Unavailability of Specialitsts Services/ Communication (system) & 0.62 & 0.15 & 2.53 & 0.5 \\ \hline
	C3\_3 - Confidence accessing healthcare remotely & Disparities in Healthcare Services & 2.44 & 1.32 & 4.53 & 0.005 \\ \hline
	D2 - Gender * & Other & 0.55 & 0.31 & 0.98 & 0.044 \\ \hline
	D7 - Living independently & Autonomy & 0.75 & 0.38 & 1.51 & 0.421 \\ \hline
	D8 - Education [ISCED] & Health Literacy & 0.82 & 0.58 & 1.18 & 0.284 \\ \hline
	D9 - Net Householdincome [per/year] & Cost of Care & 1.08 & 0.67 & 1.75 & 0.75 \\ \hline
	D10 - Financial stability & Cost of care & 2.43 & 0.97 & 6.09 & 0.059 \\ \hline
	\textsc{PD}Q-8 - \textsc{PD}Q-8 score & Self-Efficacy & 1.03 & 1.01 & 1.05 & 0.011 \\ \hline
	nPop - Population according to quantiles of German population [in sqkm] & Disparty in Health Services & 0.96 & 0.77 & 1.19 & 0.714 \\ \hline
	nGER - Neurologists nearby (per sqkm) & Disparty in Health Services & 0.97 & 0.87 & 1.07 & 0.527 \\ \hline
	vWEI - Comorbidity Index [vWEI] & Health Status & 1.08 & 0.97 & 1.21 & 0.155 \\ \hline
	%\bottomrule
	%\end{tabular}}
\end{longtable}

\begin{figure}[h!]
	\includegraphics[width=.84\textwidth]{fig4.model.comparison.v1.0.jpg}
	\caption{Comparison of the models. The full model including all 32 predictors was compatred in terms of accuracy to the reduced model resulting from the stepwise \textsc{GLM} regression. Values between both models are comparable although only 7 predictors remained in the model compared to the full model. For further details of the multilevel regression cf. Table \ref{tab2:reduced_model}}
	\label{fig4:comparison_models}
\end{figure}

\end{document}
